\subsection{App Store Optimization}
 \begin{enumerate}
     \item Create developer account.
    \begin{enumerate}
        \item Before we can publish any app on Google Play, we need to create a DEVELOPER ACCOUNT in Google Play Console.
        \item After signing up using the Google Account, we have to accept the Accept the developer Agreement
        \item After we’ve reviewed and accepted the Developer Distribution Agreement, we have to make the  payment so that we can publish our app.This is one time registration fee.
    \end{enumerate} 
    \item Create an App.
    \begin{enumerate}
        \item Now that we have set up our Play Console account, we can finally add our app. Here’s how to do that:
        \item Navigate to the ‘All applications’ tab in the menu
        \item Click on 'Create Application'
        \item Select the app’s default language from the drop-down menu
        \item Type in a title for the app
        \item Click on “Create
        \item  After we have created our app, we’ll be taken to the store entry page. Here, we will need to fill out all the details for our app’s store listing.
    \end{enumerate}
    \item Prepare store Listing
    \begin{enumerate}
        \item Before we can publish our app, we need to prepare its store listing. These are all the details that will show up to customers on our app’s listing on Google Play
        \begin{enumerate}
            \item Product Details : There are several fields that need to be filled out like :
            \begin{enumerate}
                \item App’s Title
                \item App’s description should be written with a great user experience in mind . Use the right keywords,but don’t overdo it. Make sure our app does not come across as spam-y or promotional
                or it will risk getting suspended on the Play Store.
            \end{enumerate}
            \item Graphic Assets ,Screenshots and video :
            \begin{enumerate}
                \item Under graphic assets, we can add screenshots, images, videos, promotional  graphics, and icons that showcase our app’s features and functionality. 
                \item Some parts under graphic assets are mandatory, like screenshots, a feature graphic, and a high resolution icon. Others are optional, but we can add them to make our app look more attractive to users .
                 \item There are specific requirements for each graphic asset that we upload, such as the file format and dimensions.
                \item Graphic assets, screenshots, and videos are used to highlight and promote our app on Google Play and other Google promotional channels.
            \end{enumerate}
            \item Categorization
            \begin{enumerate}
                \item This part requires us to select the appropriate type and category our app belongs to. From the drop-down menu, we can pick either app or game for the application type.
                \item There are various categories for each type of app available on the Play Store. Pick the one the app fits into best.
            \end{enumerate}
            \item Contact Details
            \begin{enumerate}
                \item This part requires us to enter contact details to offer our customers access to support regarding our app.
                \item We can add multiple contact channels here, like an email, website, and phone number, but providing a contact email is mandatory for publishing an app.
            \end{enumerate}
            \item Language Translations
            \begin{enumerate}
                \item The language set by default is English, but we can always customise this and even add translations of our app. 
                \item If we don’t include the translations, users may be seeing an automatic translation.
                \item Therefore, if the app will be available in several countries, it’s best to include a translation to ensure good user experience.
            \end{enumerate}
            \item Privacy policy
            \begin{enumerate}
                \item In this step, we must include a URL with our privacy policy so that users know how we are going to use their sensitive data and that of their device. 
                \item In addition, the link to the privacy policy must be placed both on the Play Store listing and within the app if our app requests access to sensitive data or content.
            \end{enumerate}
        \end{enumerate}
    \end{enumerate}
    \item Now before releasing our app to Google Play we have to generate Signed APK for release .
    \item Upload APK to an App Release
    \begin{enumerate}
        \item Now that we have prepared the ground to finally upload our app, it’s time to dig out our APK file.
        \item The Android Package Kit (or APK, for short) is the file format used by the Android operating system to distribute and install apps. Simply put, our APK file contains all the elements needed for our app to actually work on a device.
        \item Google offers us multiple ways to upload and release our APK. Before we upload the file, however, we need to create an app release.
        \item T o create a release, select the app we have generated. Then, from the menu on the left side, navigate to ‘Release management’ - ‘App releases.’
    \end{enumerate}
    \item Provide an Appropriate Content Rating
    \begin{enumerate}
        \item To rate our app, we need to fill out a content rating questionnaire. 
        \item To prevent  our apps from being listed as “Unrated,” sign in to Play Console and fill out the questionnaire for each of our apps as soon as possible. “Unrated” apps may be removed from Google Play.
        \item Following are the steps to add content rating :
        \begin{enumerate}
            \item Sign in to Play Console.
            \item Select an app.
            \item On the left menu, click Store presence > Content Rating.
            Review information about the questionnaire and type the email address.
            \item Click Continue.
            \item Select a category.
            \item Complete the questionnaire.
            \item On our ratings summary page, click Apply rating to my app. After we’ve applied ratings to the app, we can review ratings and questionnaires on Content Rating page. 
        \end{enumerate}
    \end{enumerate}
    \item Upload App Content: To better serve users, it's important to provide accurate information about the app. In addition to filling out the content rating questionnaire, we'll also need to provide details about our app's target audience and content. 
    \begin{enumerate}
        \item The app is designed primarily for children under 13
        \item The app is designed for everyone,including children
        \item The  app is not designed for children.
        \item Under "Target audience and content,"select Start. Fill out each section accordingly:
        \begin{enumerate}
            \item Age groups: Select the age group(s) that the app targets. We can make multiple selections.
            \item App details: We may be asked for additional details about how our app works. It is very important that we provide accurate answers to the questions as they relate to our app.
            \item Ads: If the app is serving ads to children we will be asked about it.
            \item Store presence: Apps that are primarily for children will be required to participate in Google Play’s Designed for Families program. If the app is designed for several age groups we can choose to apply to the program.
            \item Review a summary of the selections and click Confirm.
        \end{enumerate}
    \end{enumerate}
    \item Set Up Pricing \& Distribution
    \begin{enumerate}
        \item Before we can fill out the details required in this step, we need to determine the app’s monetizing strategy. 
        \item Once we know how our app is going to make money, we can go ahead and set up the app as free or paid. Remember, we can always change app from paid to free later, but we cannot change a free app to paid.
        \item For that, we’ll need to create a new app and set its price.
        \item We can also choose the countries we wish to distribute we app in, and opt-in to distribute to specific Android devices and programs too.
    \end{enumerate} 
    \item Rollout Release to Publish the App
    \begin{enumerate}
        \item We’re almost done. The final step involves reviewing and rolling out our release. Before we review and rollout our release, we make sure the store listing, content rating, and pricing and distribution sections of the app each have a green check mark next to them.
    \item Once we’re sure we’ve filled out those details,  we select our  app and navigate to ‘Release management’ –‘App releases.’ Press ‘Edit release’ next to our desired release, and review it.
    \item Next, click on ‘Review’ to be taken to the ‘Review and roll out release’ screen. Here, we can see if there are any issues or warnings we might have missed out on.
    \item Finally, select ‘Confirm roll out. This will publish our app to all users in we target countries on Google Play.
    \item If we follow all the steps above correctly, we’re all set to successfully publish our app.
    \end{enumerate}
\end{enumerate}
Now our App is in Manual Review Once the App is Approved It'll Listed On Google Play Store. The Review Time is Up to 48 hours
     

